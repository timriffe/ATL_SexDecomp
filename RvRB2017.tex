
%\documentclass[12pt,oneside,a4paper,doublespacing]{article} % for submission
\documentclass[12pt,oneside,a4paper]{article} % for sharing
% packages used for review
\usepackage{lineno}
\linenumbers
\usepackage[margin=1in]{geometry}

\usepackage{appendix}
\usepackage{amsmath}
\usepackage{caption}
\usepackage{placeins}
\usepackage{graphicx}
\usepackage{subcaption}
%\usepackage{subfig}
\usepackage{longtable}
\usepackage{setspace}
%\usepackage{tikz}
\usepackage{booktabs}
\usepackage{tabularx}
\usepackage{xcolor,colortbl}
\usepackage{chngpage}
%\usepackage[active,tightpage]{preview}
\usepackage{natbib}
%\usepackage[natbib, maxcitenames=2, mincitenames=1, style=apa]{biblatex}
\bibpunct{(}{)}{,}{a}{}{;} 
\usepackage{url}
\usepackage{nth}
\usepackage{authblk}
\usepackage[most]{tcolorbox}
%\usepackage{xcolor}
%\usepackage[usenames, dvipsnames]{color}
%\usepackage{hyperref}
%\usepackage{color}
%\usepackage{fontspec}
%\usepackage{pdfsync}
\usepackage[normalem]{ulem}
\usepackage{amsfonts}
\usepackage{xfrac}


%\renewcommand{\listtablename}{List of Appendix Tables}
%\newcolumntype{C}[1]{>{\centering\let\newline\\\arraybackslash\hspace{0pt}}m{#1}}
%\newcolumntype{L}[1]{>{\raggedright\let\newline\\\arraybackslash\hspace{0pt}}m{#1}}
% working on this need to concatenate file name based on sex and variable name
%\newcommand\Cell[1]{{\raisebox{-0.05in}{\includegraphics[height=.2in,width=.2in]{Figures/ColorCodes/\expandafter#1}}}}  

%%%%%%%%%%%%%%%%%%%%%%%%%%%%%%%%%%%%%%%%%%%%%%%%%%%%%%%%%%%%%%%%%%%%%%%%%%%%%
% setting color to letters affects spacing. Here's a hack I found here:
% http://tex.stackexchange.com/questions/212736/change-letter-colour-without-losing-letter-spacing
%\DeclareRobustCommand{\spacedallcaps}[1]{\MakeUppercase{\textsc{#1}}} % all
% caps with better spacing

%\colorlet{RED}{red}
%\colorlet{BLUE}{b}
%\colorlet{rd}{red}
%\colorlet{bl}{blue}

%%%%%%%%%%%%%%%%%%%%%%%%%%%%%%%%%%%%%%%%%%%%%%%%%%%%%%%%%%%%%%%%%%%%%%%%%%%%%%

\newcommand\ackn[1]{%
  \begingroup
  \renewcommand\thefootnote{}\footnote{#1}%
  \addtocounter{footnote}{-1}%
  \endgroup
}
%\newcommand\vt[1]{\textcolor{rd}{#1}}
%\newcommand\eg[1]{\textcolor{bl}{#1}}

%\newcommand\tg[1]{\includegraphics[scale=.5]{Figures/triadtable/triad#1.pdf}}
%\newcommand\tgh[1]{\raisebox{-.25\height}{\includegraphics[scale=.3]{Figures/triadtable/triad#1.pdf}}}

\defcitealias{HMD}{HMD}
\newcommand{\dd}{\; \mathrm{d}}
\newcommand{\tc}{\quad\quad\text{,}}
\newcommand{\tp}{\quad\quad\text{.}}
% junk for longtable caption
\AtBeginEnvironment{longtable}{\linespread{1}\selectfont}
\setlength{\LTcapwidth}{\linewidth}

%%%%%%%%%%%%%%%%%%%%%%%%%%%%%%%
\begin{document}

\title{Healthy life expectancy, mortality, and age prevalence of morbidity}

\author[1]{Tim Riffe}
\author[1]{Alyson van Raalte}
\author[1]{Maarten J. Bijlsma}
\affil[1]{Max Planck Institute for Demographic Research}

%\author{[Authors]}

\maketitle

\begin{abstract}
In calculating period healthy life expectancy, the use of age-specific morbidity
prevalence patterns assumes that age captures the important
time-variation in the given health condition, i.e. that the disabling process is related to how long an individual has lived. However, many morbidity patterns are better classified by time-to-death. At advanced ages the conflation of an increasing chronological-age mortality pattern and a time-to-death morbidity pattern produces an apparent morbidity pattern that increases with advancing age. Differences in period healthy life expectancy over time or between populations cannot easily be partitioned into morbidity and mortality components because the period morbidity pattern may depend on an unknown future time-to-death process not captured by period mortality. We illustrate these concepts formally and empirically, using morbidity data from the U.S. Health and Retirement Study.
While holding the time-to-death morbidity pattern fixed, we show that mortality
reduction alone reduces the total life years with disability. We estimate an
upper bound of bias in estimates of disability life years that are based on age
patterns of prevalence derived from different realistic time-to-death morbidity
patterns.
Our findings have implications for any between- or within-population
comparisons of period healthy life expectancy conditioned on different age patterns of mortality.
\end{abstract}

\section{Introduction}

Healthy life expectancy (HLE) is among the most widely used metrics of
population health. It combines information on mortality and morbidity to
summarize the expected years of life lived in good health, however measured. If
healthy life expectancy increases faster than life expectancy, morbidity is
compressed into a smaller proportion of life. HLE can change because of changes in mortality, morbidity, or both.

Most often, HLE is calculated on the basis of morbidity prevalence and a lifetable. In calculating HLE, the use of age-specific morbidity prevalence data implicitly
assumes that a chronological age pattern best characterizes variation in the
health characteristic over the lifespan. However research has shown that the prevalence of many health characteristics in old ages is better measured by a pattern over time-to-death (TTD) or by both age and TTD \citep{klijs2010disability,riffe2015ttd}. Most morbidity
patterns increase with age in the aggregate, and can claim empirical regularity
in this regard. However, we explain how the observed shape of a morbidity prevalence age-curve
could change due to changes in mortality even with no change in underlying morbidity. At advanced ages the conflation of an increasing chronological-age mortality pattern and a TTD morbidity pattern produces a morbidity pattern that appears to increase with advancing age. 

In the cohort perspective the TTD (here \emph{true}) and age (here \emph{apparent}) morbidity patterns imply the same HLE.
Problems arise in the period perspective. Differences in period HLE over
time or between two populations cannot easily be partitioned into morbidity and
mortality components because period morbidity patterns may depend on a TTD \emph{countdown} process already in place \citep{wolf2015disability}.\footnote{TTD prevalence patterns may also arise from age-structured health and mortality transition rates, but this is something we do not treat in detail.}
%will depend on some unknown future TTD process. 
For the same reason, comparisons of disability prevalence by age between populations with different underlying mortality may not be cut-and-dry.

In this paper we illustrate these concepts formally and empirically, using morbidity data from the U.S. Health and Retirement Study \citep{RAND, HRS}. Assuming a fixed TTD morbidity function, we show that mortality reduction alone can lead to a lower fraction of years lived in poor health, and consequently morbidity compression. We estimate the magnitude of potential biases for different disabling processes, given different levels of mortality extracted from the Human Mortality Database \citep{HMD2018}. We first explain how the age pattern of morbidity prevalence may partly be a function of mortality using both formulas and a schematic illustration.


\section{Morbidity as a function of time to death}
 \label{sec:morb}
% redraft this section, w formal part as a new subsection
 
 A joint pattern of morbidity prevalence by age and time to death could in
principle obtain any shape. However, empirically, such joint patterns appear to fall into a small set of basic types, which typically slope monotonically in a single direction over these two dimensions. For example, Figure~\ref{fig:srhpoortal} shows US female poor self-reported health (SRH) prevalence after age 70 broken down by age and TTD under three views. Panel~\ref{fig:srhpoorsurf} shows these data as a surface, whose more horizontally running prevalence contour lines indicate that time-to-death is a more important time measure to describe prevalence of poor SRH than age. Figures~\ref{fig:srhpoorttd} and~\ref{fig:srhpoorage}, show the same prevalence data plotted with respect to the respective TTD and age margins. Of these, the TTD lines in \ref{fig:srhpoorttd} yield a more compact set of patterns, which one could imaginably summarize with a single line without much loss of precision. Age lines (within TTD) in~\ref{fig:srhpoorage} are less articulated, more dispersed, and would be less efficiently described by a single age curve.  

Thus a convenience assumption of a single fixed morbidity prevalence over
time-to-death is not unrealistic, at least for characteristics with this joint
prevalence pattern. This pattern holds for many, but not all, indicators of
health and disability \citep{riffe2015ttd}. \citet{klijs2010disability} has
similarly shown that, compared to age, TTD was a stronger predictor of both
incidence and prevalence of activities of daily living disabilities, but that
functional limitations were better predicted by age.
% also severe cases more ttd than mild
Overall, there is not a broad evidence base on joint age and TTD patterns. While our simplifying assumption of fixed TTD patterns is an acceptable abstraction of the particular joint patterns from which they were derived, these patterns only serve to demonstrate an arithmetic point in the extreme. The degree to which prevalence is better measured by age or time-to-death for different disabling conditions remains an open research question.

\begin{figure}[ht!]
\makebox[\linewidth][c]{
    \centering
    \begin{subfigure}{.595\textwidth}
        \includegraphics[scale=.5]{Figures/srhpoor_f_Surf_a.pdf}
        \caption{Age by TTD}
        \label{fig:srhpoorsurf}
    \end{subfigure}
    ~
    \begin{subfigure}{.391\textwidth}
        \includegraphics[scale=.5]{Figures/srhpoor_f_TTD_b.pdf}
        \caption{TTD within Age}
        \label{fig:srhpoorttd}
    \end{subfigure}
    }
    
    \makebox[\linewidth][c]{
    \begin{subfigure}{.595\textwidth}
        \includegraphics[scale=.5]{Figures/srhpoor_f_Age_c.pdf}
        \caption{Age within TTD}
        \label{fig:srhpoorage}
    \end{subfigure}
        ~
    \begin{subfigure}{.391\textwidth}
        \includegraphics[scale=.5]{Figures/blankholder.pdf}
    \end{subfigure}
    }
    \caption{Prevalence of females from the 1920-1924 cohort with self-reported poor health, by time-to-death and age. The same data are shown by age over the time-to-death margin in (b) and by time-to-death over the age margin in (c). Comparing (b) and (c) we conclude that a single time-to-death pattern would better approximate the surface (a) than would a single age pattern.}\label{fig:srhpoortal}
\end{figure}
 
 \FloatBarrier
 
Imagine a bad health condition, $G$, with prevalence that varies only as a
function of time to death, $y$, and not as a function of chronological age, $a$.
Since the TTD prevalence distribution is very closely concentrated at the end of life, there will still be an apparent age function,
$g^\star(a)$.
In this case $g^\star(a)$ is a heterogeneous aggregate based on both mortality
and the underlying TTD process:
 
\begin{align}
g^\star(a) &= \frac{\int _0^\omega g(y) N(a,y) \dd y}{N(a)} \\
      &= \frac{\int _0^\omega g(y) N(a)
      \mu(a+y)\frac{\ell(a+y)}{\ell(a)}\dd y}{N(a)}\\
      &= \int _0^\omega g(y) f(y|a)\dd y \label{eq:gyfya}\tc
\end{align}
where $N(a)$ is the population aged $a$, $\ell(a)$ is lifetable survivorship, and
$\mu(a)$ is the force of mortality. $f(y|a)$ is the conditional remaining-years
distribution, which gives the probability of dying in $y$ years given survival
to age $a$. The expression \eqref{eq:gyfya} says that the proportion of those in
age $a$ that has condition $G$ does not depend on population structure at all, but only on future mortality rates and the TTD pattern of $G$, $g(y)$. 

A function such as $g(y)$ would have implications for the interpretation of
period age patterns of morbidity, and by extension, HLE. If a function such as $g(y)$ holds, it is tautologically true that the
measurement of HLE in completed cohorts (or stationary populations)
will be identical whether calculated on the basis of $g^\star(a)$ or the
underlying $g(y)$ pattern. Distortions only arise in the interpretation of
period HLE under changing mortality, or with period HLE comparisons between
populations with different mortality. 

%Under changing mortality and a fixed $g(y)$, the age pattern of
%morbidity, $g^\star(a)$, will change even as the morbidity process does not,
% and this is why 

Since morbidity prevalence in this scenario is partly a function of mortality,
 the age patterns of morbidity for populations with different mortality
levels or patterns cannot be compared without additional information. Under
these circumstances, it is also deceptively tricky to partition period HLE
differences into underlying morbidity and mortality components, because the
morbidity component is (arithmetically) a function of an uncertain future mortality pattern
that accompanies the apparent age pattern of morbidity. Although cohort HLE (a
gold standard) is theoretically unbiased \citep{imai2007estimation}\footnote{We confirm that this remains so even if morbidity prevalence in strongly patterned by time-to-death.}, and therefore comparable, this quantity cannot be faithfully
decomposed into morbidity and mortality components based on age patterns of morbidity and mortality alone if
the underlying morbidity pattern is a function of time-to-death.

A toy example serves to illustrate these concepts. Figure~\ref{fig:test}
provides a schematic overview of two stationary populations. The
underlying survival pattern of these two populations is based on period survival
curves from Japanese males in 1970 (a) and 2010 (b) (HMD), but the reader may
imagine these as two hypothetical populations. Population (a) has a life
expectancy of 69.3, while population (b) has a life expectancy of 79.5, slightly
more than 10 years higher. For demonstration, we partition each survival curve
into 10 lifespan quantiles, represented with horizontal bars. Our simple
TTD prevalence, $g(y)$, is drawn with identical yellow triangles at
the end of each lifespan bar. Onset begins 5 years before death and culminates with 80\%
prevalence. The chronological prevalence function is drawn with blue triangles,
with onset at age 50 reaching a maximum prevalence of 50\% at hypothetical age
111.5. Both prevalence functions are identical for populations (a) and (b). 

\begin{figure}
\makebox[\linewidth][c]{
\centering
\begin{subfigure}{.6\textwidth}
  \centering
  \includegraphics[width=.98\linewidth]{Figures/Japan1970}
  \caption{Higher mortality setting}
  \label{fig:toypop1}
\end{subfigure}%
\begin{subfigure}{.6\textwidth}
  \centering
  \includegraphics[width=.98\linewidth]{Figures/Japan2010}
  \caption{Lower mortality setting}
  \label{fig:toypop2}
\end{subfigure}
}
\caption{Schematic survival curves from higher (\ref{fig:toypop1}) and lower
(\ref{fig:toypop2}) mortality populations. Each population is
subjected to the same chronological age (blue) and time-to-death (yellow)
morbidity prevalence patterns. The total prevalence of each sums to disability life years
(DLY), drawn on the right of each survival curve. In \ref{fig:toypop1}
the age and time-to-death prevalences imply the same DLY. In
\ref{fig:toypop2} the time-to-death DLY is identical to \ref{fig:toypop1}, but
the age DLY is two years higher, due entirely to improved longevity.}
\label{fig:test}
\end{figure}

The resulting disability life years (DLY) is shown with
barplots next to each stationary population.
In population (a) the age and TTD prevalence functions yield
the same DLY (and HLE). In population (b) the time-to-death DLY is identical to
population (a), but the age DLY is nearly twice as high. For the
age prevalence function, it is correct to conclude that increased longevity leads to
increases in prevalence, but for the TTD prevalence function there is
no morbidity-mortality trade-off. Instead, improved longevity leads to increased proportions of life
lived disability-free, albeit with no change in the absolute concentration of morbidity in the final years of life. Analyses
based on the standard Sullivan method \citep{Sullivan1970} are only capable of
predicting increased DLY when projecting from the mortality of (a) to (b). This
is so for both kinds of morbidity because the TTD prevalence pattern
is erased when the same condition is measured over age. The same
Sullivan method can also only conclude that the morbidity of the TTD
process is more compressed in (b) than in (a), even though its essential
character is unchanged. Prevalence functions are in fact more nuanced than those
presented here, often varying by both age and TTD, but our example provides a useful heuristic to understand a previously-undescribed source of bias in common applications of the Sullivan method.

\section{Data and methods}
\label{sec:datamethods}
Estimating morbidity prevalence by time-to-death is straightforward for health surveys with mortality follow-up modules. We use the RAND version P of the U.S. Health and Retirement Study (HRS),\footnote{The HRS is sponsored by the National Institute on Aging (grant number NIA U01AG009740) and is conducted by the University of Michigan} which consists of 12 survey waves from 1992 through 2014 and an extensive set of measures for various aspects of morbidity. Time-to-death is calculated directly for deceased respondents as the difference between the date of death and interview date. Within-individual time-to-death trajectories of health measures are therefore directly available as such in the data. For a given binary health measure on an age by TTD grid, the 1-year age by 1-year TTD specific prevalence ($g_{a,y}$) is defined as the average value of that respective health measure in that age by TTD interval.

To ease interpretation, we produced age by TTD prevalence surfaces in which
random variation is filtered out by applying logistic regression models with a
given binary health measure as the outcome and natural splines for birth cohort,
TTD, and age as covariates. We used the predictions from these regression models
to produce a prevalence surface for each variable. To account for the
longitudinal nature of the HRS, potentially resulting in correlated error terms,
we reproduced the HRS sampling structure in a bootstrap procedure
\citep{efron1994introduction}. Annotated \texttt{R}-code for the procedure and the generation of all empirical results for this analysis is available in an online repository. The bootstrap procedure is described in a technical appendix. In a sensitivity analysis, we investigated the influence on our results of the underlying assumption that morbidity prevalence is a smooth function over birth cohort, TTD, and age; both natural splines and LOESS curves over the other time measures \citep{riffe2015ttd} produced highly similar results. We produced these surfaces for each health measure and by sex separately.
% 
% \begin{equation}
% \end{equation}


%To demonstrate how a time-to-death morbidity prevalence pattern influences bother the age pattern of morbidity to estimate healthy life expectancy we 





\section{Results}
\label{sec:ttdok}

\FloatBarrier
%\section{How changing mortality affects morbidity prevalence}
\subsection{A time-to-death morbidity pattern interacted with mortality}
\label{sec:schematic}
The age pattern of morbidity prevalence observed in the cross-section (in older ages) depends
on the extent to which prevalence is principally described by
age versus time-to-death and on the underlying mortality
level. If the prevalence is principally a function of TTD, the
specifics of its shape are also important. In Figure~\ref{fig:Fig_schematic3} we
show the age-translations of different schematic TTD prevalence patterns when
interacted with fictitious stationary populations under different mortality levels.

\begin{figure}
\begin{adjustwidth}{-1.5cm}{}
	\centering
	\includegraphics[scale=.8]{Figures/schematic3.pdf}
	\caption{The age pattern of morbidity prevalence in ages 60$+$ derived from interacting different TTD prevalence patterns (left column) with fictitious stationary populations subject to the color-coded death distributions depicted in the top right figure.}
	\label{fig:Fig_schematic3}
\end{adjustwidth}
\end{figure}

The first column contains three different types of disability, all of which are
experienced by half of the population at the time of death, but which differ in
the timing of onset prior to death and in the steepness of the curve with the
approach to death.
The first type of disability is virtually nonexistent 5 years prior to death, but then increases very rapidly as death approaches. The middle variant of disability is rare 15 years before death, but increases to about 20 percent of the population 5 years before death and rises sharply thereafter. The bottom figure depicts a disabling process that although still strictly determined by time-to-death, is common and accumulates very slowly starting from about 50 years before death.

The second column translates the time-to-death disability prevalence curves into
the ``apparent'' chronological age prevalence of disability for different
mortality levels depicted by the death density curves above. These mortality
levels roughly correspond to USA males in 2002 $e_{60} = 20.0$ years, Canadian
females in 2004 $e_{60} = 25.0$ years, and projected Japanese females a decade
or so from now $e_{60} = 30.0$ years (latest observed level in 2012 was $e_{60}
= 28.3$ years \citep{HMD2015}. In all cases increasing remaining life
expectancy results in decreasing age-specific disability prevalence by
chronological age. With steeply increasing disability prior to death (first
row), the differences in disability prevalence are largest above age 80, where
the bulk of mortality occurs, while with more gently increasing disability
(second and third rows) the greatest differences in disability age-prevalence
curves appear at younger ages.

These differences induced by mortality alone are not trivial. In the middle variant, which closely resembles the TTD prevalence of disability in bathing, a 10-year increase in $e_{60}$ results in a 50 percent drop in disability prevalence at age 80 from around 20 to 10 percent. Meanwhile, the age at which a quarter of the population were considered disabled in this scenario differed by about 5 years with a 5-year improvement in $e_{60}$ from 20 to 25.



% \begin{figure}
% \begin{adjustwidth}{-1.5cm}{}
% 	\centering
% 	\includegraphics[scale=.8]{Figures/IADL1_Males.pdf}
% 	\caption{Proportion of males from the 1915-1919 cohort with at least one (of
% 	five) instrumental activity disabilities, by time-to-death and age. A lifeline in this representation moves in a rightward descending diagonal.}
% 	\label{fig:IADL_ATL}
% \end{adjustwidth}
% \end{figure}


\FloatBarrier
\subsection{Estimating bounds for the impact of mortality differences on estimates of disabled life years}
\label{sec:bounds}
The Sullivan method is the most commonly used method to partition life
expectancy into estimates of the total life years lived in a state of good
health (HLY) or disability (DLY). Its popularity owes to its minimal data
requirements. Only current age-specific disability prevalence rates are needed
in addition to a life table or age-specific mortality rates. Specifically, the
number of person-years with disability $_{n}\pi _{x} \cdot _{n}L _{x}$ in
age-group $x$ to $x+n$ are the product of the person-years lived from the life
table $_{n}L _{x}$ and the proportion disabled $_{n}\pi _{x}$. The total DLY is
the sum of  $_{n}\pi _{x} \cdot _{n}L _{x}$ over all age groups
\citep{Sullivan1970}.

It has long been recognized that the Sullivan method does not produce a pure
synthetic measure of health expectancy, since it combines flow data of current
mortality incidence with stock data of morbidity prevalence. While the mortality
flow responds immediately to period change, for instance from medical
innovations, the morbidity stock is slower to change because it reflects past
cohort experiences with disability incidence and recovery
\citep{Mathers1997,Barendregt1994}. Moreover, as was illustrated in
Figure~\ref{fig:Fig_schematic3}, the prevalence at any age may not only depend
(mathematically)) on past transitions into and out of disability but also on
future deaths.

Nevertheless, comparing populations on the basis of life years lived in a state
of good health (HLE) or disability (DLY) is standard practice in population
health. The difference in either metric, either a within-population difference
over two time periods or a between-population difference in the same time
period, is often decomposed into mortality and morbidity components on the basis
of differences in mortality rates and morbidity age-prevalence respectively.
\citet{Andreev2002} propose an analytic decomposition of HLE based on mortality
incidence and morbidity prevalence, which partitions a difference in HLE into
two additive components for mortality and morbidity.
%According to \citet{Nusselder2004}, the
%corresponding mortality and disability effects at each age group for a
% within-population decomposition are:
%\begin{align}
%	{_{n}{MOR}_{x}}&=\left [ \frac{_{n}\pi_{x\left ( t \right )} + _{n}\pi_{x\left
%	( t+y \right )}  }{2}\right ]\cdot \Delta _{n}L_{x} \label{eq:MORcomp}\\
%	{_{n}{DIS}_{x}}&=\left [ \frac{_{n}L_{x\left ( t \right )} + _{n}L_{x\left (
%	t+y \right )}  }{2}\right ]\cdot \Delta _{n}\pi_{x}	\tc \label{eq:DIScomp}
%\end{align}
%where $\Delta$ refers to the change in the variable from time $t$ to $t+y$,
%$_{n}\pi_x$ is prevalence in the discrete age group $[x,x+n)$, and $L_x$ is
%lifetable exposure. Discrete prevalence $_{n}\pi_x$ is essentially our
%$g^\star(a)$ from equation~\eqref{eq:gyfya}. 
%The sum of the two components over
%all ages is equal to the total change in DLY. These formulas are an
%application of standard \citet{kitagawa1955components} decomposition. The
%decomposition method of \citet{Andreev2002} (not shown) yields a different and
%more correct age pattern of mortality \citep{shkolnikov2017}, but total
% mortality and disability components are equal between the two methods, so we can use either of them to make our
%point.

Although it is true that these two components arise from changes in survivorship
and morbidity prevalence respectively, difficulties arise in the interpretation
of the components as pure mortality and morbidity effects. If the time-to-death
profile of disability prevalence does not change over different mortality
regimes, inducing mortality decline alone will result in declines in the
age-pattern of prevalence component, as we demonstrated. This method of
decomposition will therefore still attribute a portion of the
change in DLY to morbidity, and we consider this to be a source of bias.

To get a sense of the upper magnitude of this bias, we test how the disability
component of this decomposition method would change when a fixed time-to-death
prevalence of morbidity is applied to different mortality regimes using empirical data. Morbidity data come from the US Health and Retirement Study while mortality and exposure data come from the Human Mortality Database \citep{HMD2015}.

Specifically, we consider the age prevalence of difficulties in carrying out at
least 1, 2, or 3 (out of 5) functional Activities of Daily Living (ADL),
difficulties in carrying out at least 1, 2, or 3 (out of 5) instrumental
Activities of Daily Living (IADL), living in a nursing home, having poor
self-rated health, and being unable to name the month of the year. We estimate
these time-to-death prevalences for quinquennial cohorts from 1905 to 1930,
separately for males and females. We calculate the time-to-death $g(y)$ profiles of
disability prevalence for all extinct lifespans above age 65 for each cohort.
Estimation is done via a loess smooth over age, time-to-death, and birth
cohorts. We
then average the time-to-death profiles of disability over lifespans and over the 6 cohorts.
% TR: no mort assumptions needed to blend thano profiles, as these are in
% principle purged of mortality.
% For more
%recent cohorts, we forecasted morbidity and mortality rates to close the cohort
%out at age 110. This was generally a small proportion of the total cohort, for
%instance from the 1915 birth cohort 6 percent of males and 11 percent of
% females were still alive. 
 The time-to-death prevalence of morbidity for
each disability type is shown for females in Figure~\ref{fig:DisbyTTD}.


\begin{figure}
\begin{adjustwidth}{-1.5cm}{}
	\centering
	\includegraphics[scale=.6]{Figures/DisbyTTD.pdf}
	\caption{The disability prevalence by time to death for various disability types, based on U.S. HRS data (females).}
	\label{fig:DisbyTTD}
\end{adjustwidth}
\end{figure}


We calculate the apparent period chronological age prevalence of morbidity,
$g^\star(a)$, for all medium to large populations of the Human Mortality
Database, had they experienced the US time-to-death profile of morbidity. Eastern European countries are excluded from this exercise due to widely varying
age patterns of mortality, particularly in the years surrounding political
transition. To calculate the age-pattern of morbidity, we assume the survival
pattern of each lifetable to be a stationary population and apply a
discretization of equation~\eqref{eq:gyfya}. 

We then make pair-wise comparisons of DLY between each population in
the same year for the years 1980, 1990, and 2000.
For within-country comparisons, we compare each population in
10-year jumps, for all years starting from 1950. Altogether this leads to
187 within-population comparisons and 1785 between-population comparisons for
each sex. Finally we decompose the change or difference in DLY between the
population pairs into mortality and morbidity components using the method of
\citet{Andreev2002}. The true value of the change in DLY and the true value of the disability component
are both zero by design. Thus, the estimated disability component from this
decomposition gauges bias.

We compare the association between the change in the estimated disability
component and the increase or difference in remaining life expectancy at age 60 for each population pair in
Figure~\ref{fig:Fig_Decomp_3x3}. By doing so we aim to provide a rough
empirically-based estimate of the upper bound of the change in the disability
component that is attributable to the different underlying mortality levels
of any two populations being compared. Thus if female $e_{60}$ increases in a
country by 5 years, up to about 1 year of the reduction in DLY that is
attributed to the disability component could be solely arising from the decrease
in mortality, in the case where disability is measured as having difficulty in
at least three ADLs. Departure from this upper bound in real life depends on
the degree to which the disability prevalence changes on a time-to-death versus chronological
age axis, the extent to which the US average time-to-death prevalence is
representative, and the departure from the stationary population assumption.

Overall, the relationship between the change in disability component and the
increase in $e_{60}$ is strikingly linear although the slopes differ for males
and females and by disability type. To some extent this is because the
final level of disability prevalence (i.e. in the final year of life) differs by
disability type.
In Appendix Figure~\ref{fig:Fig_Decomp_3x3_rel} we standardize for the maximum
disability prevalence to give a clearer comparison between disability types.
Even after standardization, the change in the disability component is
greater with larger $e_{60}$ differences for disability types with more
gradually changing prevalence levels by time-to-death and for females who
likewise have more gradually changing prevalence levels by time-to-death.
This is because steep time-to-death profiles of disability concentrate
disability prevalence near the death distribution,
whereas gradually changing (but still entirely time-to-death) disability
profiles spread out disability prevalence over a wider range of ages before
each age at death.

\begin{figure}
\begin{adjustwidth}{-1.5cm}{}
	\centering
	\includegraphics[scale=.8]{Figures/Decomp_3x3.pdf}
	\caption{Results of the hypothetical decomposition exercise. The size of the
	morbidity component using a standard decomposition method is plotted against
	the difference in remaining life expectancy at age 60 ($e_{60}$) in each
	pair of populations. Linear trend lines are also provided for each sex and decomposition type.}
	\label{fig:Fig_Decomp_3x3}
\end{adjustwidth}
\end{figure}

If projecting morbidity prevalence, one can also refer to these results as a
heuristic on the bias inherent in assuming a fixed age-pattern of morbidity.

\FloatBarrier
\section{Discussion}

Healthy life expectancy is a popular tool for analyzing population health, forming a snapshot of the hypothetical life years lived in good or poor health. This information is well-summarized in both the period and
cohort perspectives, irrespective of whether morbidity prevalence is a function of
chronological age or time to death. However, difficulties arise in the
interpretation of period differences in this quantity. The chronological age
pattern of disability can increase or decrease solely as a function of mortality
change, even when the underlying morbidity function is held constant. Thus, for
instance, observed widening ratios in the age profiles of disability prevalence
between subgroups \citep{Crimmins2001} cannot be attributed to changes in the
disabling process without taking into account changing mortality profiles. This
observation calls into question the practice of forecasting observed
age-specific rates of decline in disability \citep{Manton2006,Khaw1999}, and
especially the more common practice of holding age-patterns of disability fixed
in morbidity projections.
Health economists refer to a similar `red herring' argument, namely that medical costs are more closely associated with time-to-death than with chronological age. As a result, health
care cost projections based on a chronological age rather than time-to-death
pattern of expenditure are artificially inflated \citep{Zweifel1999,Geue2014}.

We argue that the better way to measure changes in health or disability
is from a cohort perspective \citep{Manton2000,Manton2008,Christensen2013}.
\citet{Manton2000}, for instance, found large differences between period and
cohort estimates of active life expectancy (ALE). ALE at ages 65 and 85 was
between 1.6 and 2.6 times larger in the cohort perspective than for similar
period estimates, and the expected years of life disabled were smaller in the
cohort perspective.
Additionally, they uncovered larger differences between the cohort and period
perspectives for men than women, which they attribute to differences in
disability transition rates between the sexes. The cohort-period direction of difference in ALE is consistent with a fixed TTD pattern combined with improving mortality, as of our toy example in Section~\ref{sec:morb}. We hypothesize that some of these
larger differences might also be attributable to larger mortality reduction
among men. Further, while cohort HLE estimates are unproblematic as an index,
decompositions of HLE differences between cohorts (with different mortality) into age-patterned morbidity and
mortality components would still be biased. This is because the age pattern of
morbidity is itself decomposable into morbidity and
mortality components.  

As our example illustrates, differences
in underlying mortality can lead to differences in the age profile of
disability. That mortality levels and disability prevalence
are related is perhaps not surprising. Several studies have looked at the macro relationship between overall mortality
levels and sex differences in HLE. At higher levels of life expectancy, female
advantage in healthy life expectancy diminishes, or even reverses into male
advantage \citep{vanOyen2013}. As well, the larger the proportional female
advantage in longevity, the larger the female excess in the proportion of life
in poor health \citep{Luy2014}. 
Additionally, although the association between the severity of chronic conditions and poor health was found
to be similar for men and women, morbidity prevalence rates are generally higher
among women \citep{Case2005}. It
would be worthwhile to investigate whether there might not only be differences
in the composition of chronic conditions between the sexes, but whether the
underlying morbidity pattern itself might differ between the sexes in its joint
age and time-to-death pattern \citep{riffe2015ttd}.

Empirically, not all end-of-life health conditions are exclusive functions of
time to death, but morbidity often varies as a function of
both time measures, and expressing morbidity prevalence
as a function of both age and time-to-death, $g(a,y)$, can increase precision
\citep{wolf2015disability,riffe2015ttd}.
There is great variety in the temporal variation of the prevalence of late-life
health conditions. There is also great variety in individual trajectories with
the approach to death \citep{lunney2003patterns}. 

That morbidity prevalence may for certain health conditions be patterned by
time to death does not imply that morbidity incidence is necessarily a function
of time to death. First, a sequence of health states wherein
mortality risk increases in each successive state could produce a time-to-death
prevalence pattern.\footnote{Such a minimal state model with no age patterns for state transitions or mortality risk within states can produce both an overall age pattern of mortality and a time-to-death pattern of morbidity prevalence.} Second, it is also
plausible that some morbidity conditions are linked to a more general process of dying, thereby
linking morbidity to a process that ends with death and consequently producing a
time-to-death prevalence pattern. For example, certain conditions may manifest
themselves that are not (primary) causes of the impending death but
consequences of nearness to death caused by some other (primary) factor.
Neither of these explanations conflicts with the reality that causes must
precede effects, and that therefore death cannot cause the morbidity that precedes it \citep{lynch2015commentary}.

Increasingly, data on mortality incidence and recovery are available from
multiple waves of survey panels such as the HRS, allowing researchers to
calculate healthy life expectancy using sophisticated multistate models.
Unfortunately such data is not available in all countries, or if it is, time
trends are limited to the recent past. We are not arguing that a TTD approach
should replace multistate models when such data is available. Our aim is rather
to expose the implications of comparing healthy life expectancy from
age-structured prevalence-based models with different underlying mortality
regimes.

To model prevalence as a function of
time-to-death requires no surreal understanding of how things work, but is
rather a modelling choice \citep{wolf2015disability}. When modeling for
descriptive or exploratory purposes (as we have done to produce e.g. Figure~\ref{fig:Fig_schematic3}),
and possibly for projections, one can safely use
TTD as a predictive variable.
However, using time-to-death in models intended for causal interpretation is more perilous; Some argue that TTD may function as a proxy for unobserved variables such as biomarkers for impending mortality \citep{wolf2015disability}. Others argue that including this variable as a proxy in models will introduce omitted variable bias \citep{lynch2015commentary}. Regardless, even in the descriptive case, the reader may understand a simple TTD morbidity prevalence pattern as generally hypothetical, since the current evidence base is limited in scope. Much more empirical work is needed in order to determine whether modeling morbidity
prevalence as a function of time to death is more widely applicable to
other health conditions, younger ages, more recent birth
cohorts, and other populations in different stages of epidemiological
transition. Further, even joint age-TTD prevalence patterns, such as that of Figure~\ref{fig:srhpoortal} may change over time, and are not fixed as in our examples. Nevertheless, the distortions demonstrated are likely to arise in everyday practice when comparing health trends over age between populations or over time, since many health conditions appear to show strong time-to-death components. To further test the bias we hypothesize and to design more robust synthetic measures of life lived in good or poor health, more effort should be given to data collection, mortality followup, and measurement practices.

\section{Conclusions}
We describe a likely source of bias in comparisons of healthy life expectancy or life years lived in poor health. This bias derives from a failure of age structure to capture the principal pattern of time variation in morbidity. In short, age standardization does not guarantee comparability of morbidity levels. Comparisons of period healthy life expectancy calculated from marginal age patterns of morbidity prevalence cannot be directly partitioned into morbidity and mortality effects if the joint age by time-to-death morbidity prevalence pattern varies at all by time-to-death. 

In Section~\ref{sec:morb} we show how simplistic time-to-death and age-patterns of morbidity interact with lifetable survivorship to yield opposite conclusions on morbidity compression under improved mortality. A fixed increasing age pattern of morbidity predicts a greater burden of morbidity under improved survival, but a fixed time-to-death pattern of morbidity predicts no change in the average time spent in poor health, and therefore morbidity compression. In Section~\ref{sec:schematic} we describe in greater detail how a range of schematic time-to-death morbidity patterns interact with different stationary mortality levels to produce age patterns of morbidity. In Section~\ref{sec:ttdok} we provide an example of a variable whose joint prevalence pattern over age and time-to-death is better summarized by a single time-to-death pattern than by an age pattern. We demonstrate in Section~\ref{sec:bounds} that the degree of bias in partitioning effects depends on the shape and level of time-to-death variation. 



\newpage%% this forces a new page, so that things end up a bit more where you think they should
\bibliographystyle{plainnat}
\bibliography{RvRB2017refs}

\begin{appendices}

\section{Decomposition after standardization}
In this appendix we standardize the decomposition exercise of section~\ref{sec:bounds} for the maximum
disability prevalence to give a clearer comparison between disability types.
\begin{figure}
\begin{adjustwidth}{-1.5cm}{}
	\centering
	\includegraphics[scale=.8]{Figures/Decomp_3x3_rel.pdf}
	\caption{Results of the hypothetical decomposition exercise standardized to the maximum disability prevalence for each disability type. The size of the
	morbidity component using a standard decomposition method is plotted against
	the difference in remaining life expectancy at age 60 ($e_{60}$) in each
	pair of populations. Linear trend lines are also provided for each sex and decomposition type.}
	\label{fig:Fig_Decomp_3x3_rel}
\end{adjustwidth}
\end{figure}


\section{Description of bootstrapping technique}
The Health and Retirement Survey (HRS) data is a longitudinal dataset, and hence
observations are not independent but correlated within individuals. To account
for the influence of the correlated error structure on the estimation of age
by time-to-death (TAL) surfaces, we applied a bootstrap
\citep{efron1994introduction} procedure in which we approximated the sampling structure of the HRS. In our
application, bootstrap sizes of 999 provided the best approximation of the distribution of our estimator; if our method is adapted to other datasets, we recommend the use of bootstrap diagnostics applied to various locations on the TAL surface (with special attention to the ``edges'' of the TAL surface) to determine bootstrap size.

\subsection{Bootstrap steps}
An \texttt{R}-code application of our method is available as Supplemental Material. The method, and therefore the code, works through the following 9 steps:
\begin{enumerate}
\item{Choose an outcome variable (i.e. BMI, hospital stay duration, etc.).}
\item{Create an empty dataset with time dimensions of interest (in our application: birth year, time to death, and chronological age, but this could be more dimensions if desired), which we call the Lexis space}
\item{Resample individuals with replacement.}
\item{Create natural cubic splines for the dimensions of the Lexis space.}
\item{Fit regression model of a type appropriate to the outcome variable to the resampled data with the natural cubic splines (step 4) as covariates.}
\item{Use the regression fit to predict the Lexis space and save these predictions.}
\item{Repeat steps 3 to 6 until desired bootstrap size is achieved.}
\item{Take the median or mean over the bootstrap iterations for each Lexis cell to create a Lexis space with median or mean estimates.\begin{itemize}\item{Optional: take $\alpha/2$ and  $1-\alpha/2$ quantiles over the bootstrap iterations for each Lexis cell in order to create Lexis spaces containing the lower and upper $1-\alpha$ confidence bounds of the estimates.}\end{itemize}}
\item{Take a cross-section of the Lexis space with median or mean predictions in order to plot the Lexis surface.}
\end{enumerate}

The rest of this appendix sheds more light on these steps. See also the supplementary R-code for the implementation of the method.

\subsection{Resampling with replacement and probability weights}
The HRS samples individuals each year. Sampled individuals stay in the sample until either death or drop-out. To account for the changing population structure of the United States, probability weights for an individual can differ between observations (i.e. they change over time).

In order to account for the longitudinal nature of the data, we resample individuals following the sampling structure of the HRS. A (re)sampled individual contributes all of his or her observations to that respective (re)sample. This type of resampling is also known as blockwise bootstrapping in a time series context \citep{buhlmann1995blockwise}. The probability for an individual to be sampled is equal to the HRS probability weight of the first observation of that individual (RAND variable \texttt{wtresp}). We resample individuals with replacement. The number of individuals resampled is equal to the number of individuals in the original sample.

The information contained in the probability weights of subsequent observations of individuals is also used: We create new weights termed rescale weights, which are each individual’s probability weights divided by that individual’s respective sampling probability (and hence, the rescale weight on the first observation for each individual is always 1). These rescaled weights are used in the regression procedure (see section~\ref{app:splinesreg}), where the sampling probability represents the representativeness of the individual in their respective demographic category, the rescale weights capture to what extent this weight changes as a consequence of changes in population structure, and therefore captures to what extent an individual’s representativeness changes. In our application of the method, the vast majority of resampling weights did not exceed the 0.90 to 1.10 range, i.e. the sampling probability was never inflated or deflated by more than 10 percent.

\subsection{Regression natural cubic splines}
\label{app:splinesreg}

To produce smooth age by TAL surfaces we applied regression models to the
outcome of interest (BMI, hospital stays, etc.). Covariates in these regression models were natural cubic splines of the dimensions age, birth year, and time to death. The cubic spline is defined as follows:

\begin{equation}
S(x) = \begin{cases}
    S_0(x) = a_0x^3+b_0x^2+c_0x+d_0 &\quad t_0 \le x \le t_1 \\
    \quad\quad\quad\quad\quad\quad \vdots \\
    S_{k-1}(x) = a_{k-1}x^3 +b_{k-1}x^2+c_{k-1}x+d_{k-1} &\quad t_{k-1} \le x \le t_k        \quad ,   
\end{cases}
\end{equation}
where $x$ represents points on the dimension (in our application, a time dimension) of interest, and $t_0$ and $t_k$  denote the two extremes (endpoints) of that part of the dimension of interest for which we wish to create this spline. Furthermore, the constraints
\begin{align}
S_{k-1}(x_i) =& S_k(x_i)\\
S'_{k-1}(x_i) =& S'_k(x_i)\\
S''_{k-1}(x_i) =& S''_k(x_i)
\end{align}
with $i=1,2,\ldots k-1$ must be satisfied. This ensures that the segments of the spline meet at the boundaries (first constraint) in a smooth manner (second and third constraint). The additional constraint
\begin{equation}
S''_{k-1}(x_i) = S''_k(x_i)
\end{equation}
makes it into a natural cubic spline, i.e the second derivative of each polinomial is set to 0 at the endpoints of our part of the time dimension of interest.

The part of the dimension of interest that can be found between $t_0$ and $t_k$ is subdivided into $k-1$ segments, the connection of these segments are known as knots. In our application, knots for birth year were placed at the year 1902.5 to 1925.5 at 5-year intervals. Knots for time to death were placed at 0.5, 1, 2, 4, 7.5 and 10 years to death. Knots for chronological age were placed at 72.5 to 97.5 years of age at 5 year intervals. We denote the natural cubic spline for age as $A(x)$, for birth cohort as $C(x)$ and for time to death as T$(x)$.

For continuous outcomes, we used these splines in linear regression models, i.e.
\begin{equation}
\mathbb{E}[Y| x_a,x_c,x_t]=A(x_a)+C(x_c)+T(x_t) \quad,
\end{equation}
where $Y$ is the outcome variable of interest, and $x_a,x_c,x_t$ represent (observed) points on the age, birth year and time to death axes, respectively. For binary outcomes, we used logistic regression, i.e.
\begin{equation}
logit\{\mathbb{E}[Y| x_a,x_c,x_t]\}=A(x_a)+C(x_c)+T(x_t) \quad .
\end{equation}

Ordinal outcomes were dichotomized during data handling and therefore also modeled using logistic regression.

For count outcomes we used zero-inflated Poisson regression, which is a mixture model, combining logistic and Poisson regression. This type of model separates modeling the probability that an observation is an ``excess'' 0 (or not) from the ``count'' component (which is modeled if the observation is not an excess 0). The count component was therefore modeled as:
\begin{equation}
ln\{\mathbb{E}[Y| x_a,x_c,x_t]\}=A(x_a)+C(x_c)+T(x_t) \quad .
\end{equation}

The 0 or not component was modeled with logistic regression as denoted above. We
used this type of model because most of our observations contained a very large
number of zeros relative to observed counts and therefore appeared to come from
a mixture distribution. Ordinary Poisson models can be compared with
zero-inflated Poisson models using the Vuong test, but this was omitted in our
application since most outcomes were very clearly zero-inflated and furthermore
zero-inflated Poisson models can also adequately model non-zero inflated count
outcomes (with some loss of statistical precision, but significance or
confidence intervals are not the focus of this work).
Most count variables in our dataset were bounded, e.g. the number of nights that can be spent in the hospital is bounded by 365 (maximum number of days in a year). In those instances, we included those bounds in our Poisson model through an offset term:

\begin{align}
ln\{\mathbb{E}\left[\frac{Y}{max~bound} |  x_a,x_c,x_t\right]\}&=A(x_a)+C(x_c)+T(x_t) \\
\intertext{$\implies $}\notag 
ln\{\mathbb{E}[ Y|  x_a,x_c,x_t] \} &= A(x_a)+C(x_c)+T(x_t)+ln(max~ bound) \notag \quad.
\end{align}

Since the functions $A(x)$, $C(x)$ and $T(x)$ are cubic natural splines, they represent a vector of coefficients corresponding to $S(x)$ as defined earlier. Specification of our models in this manner with natural splines was inspired by work of \citet{carstensen2007age}.

Because the three time dimensions included in our models are not linearly
dependent, additional constraints on one dimension (such as required in APC
modeling, see e.g. \citet{clayton1987models}) were not required. We did not exhaustively investigate the application of splines along different time dimensions, or such splines in addition to splines of the time dimensions already present. However, exploratory results indicated that this would not have substantially affected our conclusions. Similarly, a higher number of knots had only minor effects on the smoothed TAL surfaces. Finally, we also compared our natural cubic splines to other flexible curve fitting methods (such as LOESS and GAM), and found natural cubic splines to provide the best fit overall to the empirical data, as expressed in distance to squared and absolute residuals and propensity for systematic under- or over-estimation of the data along relevant time dimensions (including edge effects).

Splines are known to function less well near the endpoints ($t_0$ and $t_k$); to determine if this was present in our data, we investigated the residuals of regression models. We found only weak evidence for the existence of edge effects, and for most outcomes edge effects were absent. For those outcomes where edge effects were potentially present, as a sensitivity analysis, in our resampling procedure we purposely oversampled observations near the edges and estimated TAL surfaces under this regime. We did not find this to substantially affect our conclusions. 

\subsection{Predicting the age by time-to-death surface}
To produce age by time-to-death surfaces, we created a dataset with birth year,
time to death, and chronological age as columns. The minimum and maximum values of birth year, time to death and chronological age were equal to those found in the empirical data, and intermediate values were spaced at 1-year intervals. This dataset therefore represents an empty Lexis space. Using the fitted regression models, we then predicted outcomes for this Lexis space; i.e. we created a column with an expected value for each birth year, time to death and chronological age combination (henceforth: Lexis cell) in the Lexis space. Each iteration of the bootstrap added a new column of such predictions. Note that if the user also wishes to produce standard errors (i.e. statistical precision is of importance), techniques such as multiple predictions for a Lexis cell (and averaging over them) within each bootstrap iteration could be applied to reduce Monte Carlo error. Once all iterations are finished, to produce the final Lexis surface we then plot the median prediction for each Lexis cell. Since three dimensions are more difficult to visualize due to overplotting, in our paper we chose the predictions of the 1915-1919 birth cohort and plotted along the time to death and age dimensions.

We found that, overall, predictions for each Lexis cell were normally
distributed, and TAL surfaces based on Lexis cell averages were nearly identical
to those for TAL surfaces based on median values. However, since some cells had more skewed predictions, namely when the outcome was binomial and predictions were close to 0 or 1 (the edge of the parameter space), we chose the median as the appropriate measure of centrality. Additionally, the median, being the 50\% quantile, is also consistent with 95\% quantile bootstrap confidence intervals. As their name implies, 95\% quantile bootstrap confidence intervals can be calculated by taking the 2.5\% and 97.5\% quantile for each Lexis cell and plotting those surfaces. The value of the median will, by definition, not exceed the 2.5\% and 97.5\% quantiles, whereas in theory the mean could exceed these confidence bounds in highly skewed distributions.
\end{appendices}

\end{document}