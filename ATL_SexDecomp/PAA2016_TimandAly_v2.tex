% manuscript, to be drafted

%%This is a very basic article template.
%%There is just one section and two subsections.
%\documentclass[12pt,oneside,a4paper,doublespacing]{article} % for submission
\documentclass[11pt,oneside,a4paper]{article} % for sharing

\usepackage{appendix}
\usepackage{amsmath}
\usepackage{caption}
\usepackage{placeins}
\usepackage{graphicx}
\usepackage{subcaption}
%\usepackage{subfig}
\usepackage{longtable}
\usepackage{setspace}
%\usepackage{tikz}
\usepackage{booktabs}
\usepackage{tabularx}
\usepackage{xcolor,colortbl}
\usepackage{chngpage}
%\usepackage[active,tightpage]{preview}
\usepackage{natbib}
\bibpunct{(}{)}{,}{a}{}{;} 
\usepackage{url}
\usepackage{nth}
\usepackage{authblk}
\usepackage[most]{tcolorbox}
\usepackage{xcolor}
%\usepackage{hyperref}
%\usepackage{color}
%\usepackage{fontspec}
%\usepackage{pdfsync}
\usepackage[normalem]{ulem}
\usepackage{amsfonts}
\usepackage{xfrac}
%\renewcommand{\listtablename}{List of Appendix Tables}
%\newcolumntype{C}[1]{>{\centering\let\newline\\\arraybackslash\hspace{0pt}}m{#1}}
%\newcolumntype{L}[1]{>{\raggedright\let\newline\\\arraybackslash\hspace{0pt}}m{#1}}
% working on this need to concatenate file name based on sex and variable name
%\newcommand\Cell[1]{{\raisebox{-0.05in}{\includegraphics[height=.2in,width=.2in]{Figures/ColorCodes/\expandafter#1}}}}  

%%%%%%%%%%%%%%%%%%%%%%%%%%%%%%%%%%%%%%%%%%%%%%%%%%%%%%%%%%%%%%%%%%%%%%%%%%%%%
% setting color to letters affects spacing. Here's a hack I found here:
% http://tex.stackexchange.com/questions/212736/change-letter-colour-without-losing-letter-spacing
%\DeclareRobustCommand{\spacedallcaps}[1]{\MakeUppercase{\textsc{#1}}} % all
% caps with better spacing

%\colorlet{RED}{red}
%\colorlet{BLUE}{b}
%\colorlet{rd}{red}
%\colorlet{bl}{blue}

%%%%%%%%%%%%%%%%%%%%%%%%%%%%%%%%%%%%%%%%%%%%%%%%%%%%%%%%%%%%%%%%%%%%%%%%%%%%%%

\newcommand\ackn[1]{%
  \begingroup
  \renewcommand\thefootnote{}\footnote{#1}%
  \addtocounter{footnote}{-1}%
  \endgroup
}
%\newcommand\vt[1]{\textcolor{rd}{#1}}
%\newcommand\eg[1]{\textcolor{bl}{#1}}

%\newcommand\tg[1]{\includegraphics[scale=.5]{Figures/triadtable/triad#1.pdf}}
%\newcommand\tgh[1]{\raisebox{-.25\height}{\includegraphics[scale=.3]{Figures/triadtable/triad#1.pdf}}}

\defcitealias{HMD}{HMD}
\newcommand{\dd}{\; \mathrm{d}}
\newcommand{\tc}{\quad\quad\text{,}}
\newcommand{\tp}{\quad\quad\text{.}}
% junk for longtable caption
\AtBeginEnvironment{longtable}{\linespread{1}\selectfont}
\setlength{\LTcapwidth}{\linewidth}

%%%%%%%%%%%%%%%%%%%%%%%%%%%%%%%
\begin{document}

\title{Accounting for temporal variation in morbidity measurement
and projections}

\author[1]{Alyson van Raalte\thanks{Authors are currently ordered in reverse alphabetical order. Correspondence to vanraalte@demogr.mpg.de or tim.riffe@gmail.com}}
\author[1]{Tim Riffe}
\affil[1]{Max Planck Institute for Demographic Research}

%\author{[Authors]}

\maketitle

\begin{abstract}
In calculating healthy life expectancy, our use of age-specific morbidity prevalence data implicitly assumes a chronological age at onset, i.e. that the disabling process is related to how long an individual has lived. However, many common disabling processes are better measured by a thanatological (time-to-death) age pattern. Nevertheless, because death occurs most often at advanced ages, the conflation of a chronological mortality pattern and a thanatological morbidity age pattern will produce an apparent morbidity pattern that also increases with advancing age. In the cohort perspective the true and apparent morbidity patterns will be the same. The problems arise in the period perspective. Comparing the changes in period HLE over time cannot easily be partitioned into morbidity and mortality components, because the period morbidity component will depend on some unknown future time-to-death process. In this paper we illustrate these concepts formally and empirically, using morbidity data from the Health and Retirement Survey. While holding the time-to-death morbidity function fixed, we show that mortality reduction alone reduces the total life years with disability (DLY). We estimate the magnitude of this bias for different disabling processes. This has implications for any between- or within-population comparisons of period HLE with different age patterns of mortality. 
\end{abstract}

\newpage
\section{Introduction}

Healthy life expectancy (HLE) is among the most widely used metrics of population health. It combines information on mortality and morbidity to summarize the expected years of life lived in good health, however measured. If healthy life expectancy is increasing faster than life expectancy, morbidity is being compressed into a smaller proportion of life. HLE can increase because of changes mortality, morbidity or both. 

In calculating HLE, our use of age-specific morbidity prevalence data implicitly assumes a chronological age at onset, i.e. that the disabling process is related to how long an individual has lived. However, \citet{riffe2015ttd} have shown that many common disabling processes are better measured by a thanatological (time-to-death) age pattern. Nevertheless, because death occurs most often at advanced ages, the conflation of a chronological mortality pattern and a thanatological morbidity age pattern will produce an apparent morbidity pattern that also increases with advancing age. In the cohort perspective the true and apparent morbidity patterns will be the same. The problems arise in the period perspective. Comparing differences in period HLE between two populations cannot easily be partitioned into morbidity and mortality components, because the period morbidity component will depend on some unknown future time-to-death process. For the same reason, comparisons of disability prevalence rates by age are not recommended between populations with different underlying age schedules of mortality.

In this paper we illustrate these concepts formally and empirically, using morbidity data from the Health and Retirement Survey. While assuming a fixed time-to-death morbidity function, we show that mortality reduction alone reduces the total life years with disability (DLY). We estimate the magnitude of this bias for different disabling processes, given different levels of mortality. 


\section{Morbidity as a function of time-to-death}
 
Imagine a bad health condition, $G$, with prevalence that varies as a
function of time-to-death, $y$, and not as a function of chronological age, $a$.
Since the distribution of time to death in older ages is regular and law-abiding, there will still be an
apparent age function, $g^\star(a)$ (the Sullivan curve).
In this case $g^\star(a)$ is a heterogenous aggregate based on both mortality
and the real underlying time-to-death process:
 
\begin{align}
g^\star(a) &= \frac{\int _0^\omega g(y) N(a,y) \dd y}{N(a)} \\
      &= \frac{\int _0^\omega g(y) N(a)
      \mu(a+y)\frac{\ell(a+y)}{\ell(a)}\dd y}{N(a)}\\
      &= \int _0^\omega g(y) f(y|a)\dd y \label{eq:gyfya}\tc
\end{align}
where $N(a)$ is the population aged $a$, $\ell(a)$ is the survival function, and
$\mu(a)$ is the force of mortality. $f(y|a)$ is the conditional remaining-years
distribution, which gives the probability of dying in $y$ years given survival
to age $a$. In this way, the expression \eqref{eq:gyfya} is purged of population
structure.
That is, the population of age $a$ that has condition $G$ does not depend on population structure at all, but only
on future mortality rates and the time-to-death pattern of $G$, $g(y)$. 

A function such as $g(y)$ would have implications for the interpretation of
period age patterns of morbidity, and by extension, HLE. If a function such as $g(y)$ holds, it is tautologically true that the
measurement of HLE in completed cohorts (or stationary populations)
will be identical whether calculated on the basis of $g^\star(a)$ or the
underlying $g(y)$ pattern. Distortions only arise in the interpretation of
period HLE under changing mortality, or with period HLE comparisons between
populations with different mortality. 

%Under changing mortality and a fixed $g(y)$, the age pattern of
%morbidity, $g^\star(a)$, will change even as the morbidity process does not,
% and this is why 

Since morbidity prevalence in this scenario is partly a function of mortality,
 the age patterns of morbidity for populations with different mortality
levels or patterns cannot be compared without additional information. Under
these circumstances, it is also deceptively tricky to partition period HLE
differences into underlying morbidity and mortality components, because the
morbidity component is a function of an uncertain future mortality pattern
that determines the apparent age pattern of morbidity. Although cohort HLE (a
gold standard) is theoretically unbiased \citep{imai2007estimation}\footnote{We confirm that this remains so even if morbidity prevalence in strongly patterned by time-to-death.}, and therefore comparable, this quantity cannot be faithfully
decomposed into morbidity and mortality components based on age patterns of morbidity and mortality alone if
the underlying morbidity pattern is a function of time-to-death.

That morbidity prevalence may for certain health conditions be a function of
time to death does not imply that morbidity incidence is necessarily a function of time to death. An
age-patterned sequence of health states wherein mortality risk increases with
passing states could also produce a time-to-death prevalence pattern. It is also
plausible that some morbidity conditions are linked to a more general process of
dying, thereby linking morbidity to a process that ends with death and
consequently producing a time-to-death prevalence pattern. Either of
these explanations circumvents the temptation to state that causes must
precede effects, and that therefore death cannot cause the morbidity that
precedes it. To model prevalence as a function of time-to-death requires no
surreal understanding of how things work, but is rather a modelling choice.
 
We first illustrate these concepts with a toy example. Figure~\ref{fig:test}
provides a schematic overview of two stationary populations. The
underlying survival pattern of these two populations is based on period survival
curves from Japanese males in 1970 (a) and 2010 (b) (HMD), but the reader may
imagine these as two hypothetical populations. Population (a) has a life
expectancy of 69.3, while population (b) has a life expectancy of 79.5, slightly
more than 10 years higher. For demonstration, we partition each survival curve
into 10 lifespan quantiles, represented with horizontal bars. Our simple
time-to-death prevalence, $g(y)$, is drawn with identical yellow triangles at
the end of each lifespan bar. Onset begins 5 years before death and culminates with 80\%
prevalence. The chronolgical prevalence function is drawn with blue triangles,
with onset at age 50 reaching a maximum prevalence of 50\% at hypothetical age
111.5. Both prevalence functions are identical for populations (a) and (b). 

\begin{figure}
\centering
\begin{subfigure}{.5\textwidth}
  \centering
  \includegraphics[width=.98\linewidth]{Japan1970}
  \caption{Higher mortality setting}
  \label{fig:toypop1}
\end{subfigure}%
\begin{subfigure}{.5\textwidth}
  \centering
  \includegraphics[width=.98\linewidth]{Japan2010}
  \caption{Lower mortality setting}
  \label{fig:toypop2}
\end{subfigure}
\caption{Schematic survival curves from higher (\ref{fig:toypop1}) and lower
(\ref{fig:toypop2}) mortality populations. Each population is
subjected to the same chronological age (blue) and time-to-death (yellow)
morbidity prevalence patterns. The total prevalence of each sums to disability life years
(DLY), drawn on the right of each survival curve. In \ref{fig:toypop1}
the age and time-to-death prevalences imply the same DLY. In
\ref{fig:toypop2} the time-to-death DLY is identical to \ref{fig:toypop1}, but
the age DLY is two years higher, due entirely to improved longevity.}
\label{fig:test}
\end{figure}

The resulting disability life years (DLY) is shown with
barplots next to each stationary population.
In population (a) the chronological and time-to-death prevalence functions yield
the same HLE. In population (b) the time-to-death DLY is identical to population
(a), but the chronological DLY nearly doubled. For the chronological prevalence
function, it is correct to conclude that increased longevity leads to
increases in prevalence, but for the time-to-death prevalence function there is
no morbidity-mortality trade-off. Instead, improved longevity leads to increased
proportions of life lived disability-free. Analyses based on the standard Sullivan method are only
capable of coming to the latter conclusion. In real life, prevalence functions
are more complex, but our example provides a useful heuristic to understand a
previously-undescribed source of bias in common applications of the Sullivan
method.

\section{How changing mortality affects morbidity prevalence}

The age pattern of disability prevalence observed in the cross-section depends on the extent to which the disabling prevalence is better modeled by years lived versus time-to-death, the shape of the disabling time-to-death prevalence curve, and the underlying mortality level. The effect of these latter two dimensions are shown on a fictitious population in Figure~\ref{fig:Fig_schematic3}, where for simplicity, we assume a disabling prevalence that is completely determined by thanatological age and stationary populations. 

\begin{figure}
\begin{adjustwidth}{-1.5cm}{}
	\centering
	\includegraphics[scale=.8]{schematic3.pdf}
	\caption{Schematic...}
	\label{fig:Fig_schematic3}
\end{adjustwidth}
\end{figure}

The first column contains three different types of disability, all of which are experienced by half of the population at the time of death, but which differ in the timing of onset prior to death. The first type of disability is virtually nonexistent 5 years prior to death, but then increases very rapidly as death approaches. The middle variant of disability is rare 15 years before death, but increases to about 20 percent of the population 5 years before death and rises sharply thereafter. The bottom figure depicts a disabling process that although still strictly determined by time-to-death, is common and accumulates very slowly starting from about 50 years before death. 

The second column translates the thanatological age prevalence curves of disability into chronological age prevalence of disability, for different mortality levels depicted by the death density curves above. These mortality levels roughly correspond to USA males in 2002 $e_{60} = 20.0$ years, Canadian females in 2004 $e_{60} = 25.0$ years, and projected Japanese females a decade or so from now $e_{60} = 30.0$ years (latest observed level in 2012 was $e_{60} = 28.3$ years \citep{HMD2015}).  In all cases increasing remaining life expectancy results in decreasing age-specific disability prevalence by chronological age.  With steeply increasing disability prior to death (first row), the differences in disability prevalence are largest above age 80, where the bulk of mortality occurs, while with more gently increasing disability (second and third rows) the differences in disability prevalence curves appear at younger ages. 

These differences induced by mortality change alone are not trivial. In the middle variant, which closely resembles the thanatological age prevalence of disability in bathing, a 10-year increase in $e_{60}$  resulted in a 50 percent drop in disability prevalence at age 80 from around 20 to 10 percent. Meanwhile, the age at which a quarter of the population were considered disabled in this scenario differed by about 5 years with a 5-year improvement in $e_{60}$ from 20 to 25.
Of course the disability prevalence is rarely a function of thanatological age alone. For example, Figure~\ref{fig:Fig_ADL_thana-chrono_rev} \textcolor{red}{should change this figure to the new way of calculating ADL} shows US male ADL-disability prevalence as a function of chronological and thanatological age. The horizontally running gradients indicate that time-to-death is a more important time axis to predict deterioration in ADL than time since birth. Thus our assumption of fixed morbidity prevalence on the thanatological age axis is not altogether unrealistic. This is the case for many, but not all, indicators of health and disability \citep{riffe2015ttd}. Thus in reality, the prevalence of disability will vary on both the chronological and thanatological age axes. The degree to which prevalence is better measured on a chronological versus thanatological axes for different disabling conditions remains an open research question.


\begin{figure}
\begin{adjustwidth}{-1.5cm}{}
	\centering
	\includegraphics[scale=.8]{Fig_ADL_thana-chrono_rev.pdf}
	\caption{...}
	\label{fig:Fig_ADL_thana-chrono_rev}
\end{adjustwidth}
\end{figure}



\section{Estimating bounds for the impact of mortality differences on estimates of disabled life years}

The Sullivan method is the most commonly used method to partition life expectancy into estimates of the total life years lived in a state of good health (HLY) or disability (DLY). Its popularity is owing to its minimal data requirements. Only current age-specific disability prevalence rates are needed in addition to a life table or age-specific mortality rates. Specifcally, the number of person-years with disability $_{n}\pi _{x} \cdot _{n}L _{x}$ in age-group $x$ to $x+n$ are the product of the person-years lived from the life table $_{n}L _{x}$ and the proportion disabled $_{n}\pi _{x}$. The total DLY is the sum of  $_{n}\pi _{x} \cdot _{n}L _{x}$ over all age groups.

It has long been recognized that the Sullivan method does not produce a pure synthetic measure of health expectancy, since it combines flow data of current mortality incidence with stock data of morbidity prevalence. While the mortality flow responds immediately to period change, for instance from medical innovations, the morbidity stock is slower to change because it reflects past cohort experiences with disability incidence and recovery \citep{Mathers1997,Barendregt1994}. Moreover, as was illustrated in Figure~\ref{fig:Fig_schematic3}, the prevalence stock at any age depends not only on past flows into and out of disability but also on future flows into death. \textcolor{red}{Does this make sense to say?}

Nevertheless, comparing populations on the basis of life years lived in a state of good health (HLY) or disability (DLY) is standard practice in population health. The difference in either metric, either a within-population difference over two time periods or a between-population difference at the same time period, is often decomposed into mortality and morbidity components on the basis of changes in survivorship and morbidity prevalence respectively \citep{Nusselder2004,Andreev2002}. According to \citet{Nusselder2004}, the corresponding mortality and disability effects at each age group for a within-population decomposition are:

\begin{equation}\label{eq:MORcomp}
	{_{n}{MOR}_{x}}=\left [ \frac{_{n}\pi_{x\left ( t \right )} + _{n}\pi_{x\left ( t+y \right )}  }{2}\right ]\cdot \Delta _{n}L_{x}.
\end{equation}

\begin{equation}\label{eq:DIScomp}
	{_{n}{DIS}_{x}}=\left [ \frac{_{n}L_{x\left ( t \right )} + _{n}L_{x\left ( t+y \right )}  }{2}\right ]\cdot \Delta _{n}\pi_{x}.	
\end{equation}

where $\Delta$ refers to the change in the variable from time $t$ to $t+y$. The sum of the two components over all ages equal the total change in DLY.

Although it is true that these two components arise from changes in survivorship and morbidity prevalence respectively, difficulties arise in the interpretation of the components as pure mortality and disability effects. If the thanatological age profile of disability prevalence does not change over different mortality regimes, inducing mortality decline alone will result in declines in the disability component.

To get a sense of the upper magnitude of this bias, we tested how the disability component would change when a fixed thanatological age prevalence of morbidity was applied to different mortality regimes using empirical data. Morbidity data came from the US Health and Retirement Study while mortality and exposure data came from the Human Mortality Database \citep{HMD2015}.

Specifically, we considered the age prevalence of difficulties in carrying out at least 1, 2, or 3 functional Activities of Daily Living (ADL), difficulties in carrying out at least 1, 2, or 3 instrumental Activities of Daily Living (IADL), living in a nursing home, having poor self-rated health, and being unable to name the month of the year. We calculated these age prevalences for cohorts 1905 to 1930, separately for males and females. Because these cohorts were extinct or near-extinct, we were able to calculate the thanatological $g(y)$ age profiles of disability prevalence, borrowing information from neighboring cohorts to obtain smooth age profiles using a Loess filter. For more recent cohorts, we forecasted morbidity and mortality rates to close the cohort out at age 110. This was generally a small proportion of the total cohort, for instance from the 1915 birth cohort 6 percent of males and 11 percent of females were still alive. Finally we averaged our thanatological age profiles of disability over the 6 cohorts. The thanatological prevalence of morbidity for each disability type is presented in \textcolor{red}{Figure to be made}.

In the next step we calculated the apparent period chronological age prevalence of morbidity, $g^\star(a)$, for all medium to large populations of the Human Mortality Database, had they experienced the American thanatological age profile of morbidity. To do so, we assumed that the population behind each life table was stationary. This means that for each age group, we created a vector of the distribution of remaining lifespans by time to death which we then multiplied by the corresponding disability prevalence by thanatological age. \textcolor{red}{Is this clear or would it be better to put in equations?}

Eastern European countries were excluded because of their very different age patterns of mortality and morbidity, particularly in the years surrounding political transition. We then made pair-wise comparisons of DLY in each population. For within-country comparisons, we compared each population 10 years apart, in all years starting from 1950 and ending in 0. For between-country comparisons, we compared each population with another in the same year, in years 1980, 1990 and 2000. Altogether this led to 187 within-population comparisons and 1785 between-population comparisons for each sex. Finally we decomposed the change or difference in DLY between the population pairs into mortality and morbidity components using the \citet{Nusselder2004} method described above.

We compared the association between the change in the disability component (equation \ref{eq:DIScomp}) and the increase or difference in remaining life expectancy at age 60 for each population pair in Figure~\ref{fig:Fig_Decomp_3x3}. By doing so we aim to provide a rough empirically-based estimate of the upper bound of the change in the disability component that is attributable to the different underlying mortality schedules of any two populations being compared. Thus if female $e_{60}$ increases in a country by 5 years, up to about 1 year of the reduction in DLY that is attributed to the disability component could be solely arising from the decrease in mortality, in hte case where disability is measured as having difficulty in at least three ADLs. Departure from this upper bound depends on the degree to which the disability prevalence changes on a thanatological versus chronological age axis, the extent to which the American average thanatological age prevalence is applicable to the populations being compared, and the departure from the stationary population assumption.

Overall, the relationship between the change in disability component and the increase in $e_{60}$ is strikingly linear although the slopes differ for men and women and by disability type. To some extent this is because the final level of disability prevalence (i.e. at death or thanatological age equal to 0) differs by disability type. In Appendix Figure~\ref{fig:Fig_Decomp_3x3_rel} we standardized for the maximum disability prevalence to give a clearer comparison between disability types. Even in the standardized case, however, the change in disability component was greater with larger $e_{60}$ differences for disability types with more gradually changing prevalence levels by thanatological age and for women who likewise have more gradually changing prevalence levels by thanatological age. This is because with a steep thanatological age profile of disability the prevalence rates will approach the death distribution in the chronological axis, while with a more gradually changing but still entirely thanatological age profile of disability the prevalence rates will begin to change at ages farther from the actual ages at death \textcolor{red}{Does this make sense?}.


\begin{figure}
\begin{adjustwidth}{-1.5cm}{}
	\centering
	\includegraphics[scale=.8]{Decomp_3x3.pdf}
	\caption{...}
	\label{fig:Fig_Decomp_3x3}
\end{adjustwidth}
\end{figure}




\section{Discussion}

Healthy life expectancy remains a popular tool for analyzing population health. At any given time, a snapshot of the life years lived in good or poor health are captured. This information is accurately summarized in both the period and cohort perspective, regardless of whether morbidity is a function of chronological or thanatological age. The difficulties arise in the interpretation of period differences in this quantity. The chronological age pattern of disability can increase or decrease solely as a function of mortality change even when the underlying morbidity function is held constant. Thus, for instance, observed widening ratios in the age profiles of disability prevalence between subgroups \citep{Crimmins2001} cannot be attributed to changes in the disabling process without taking into account changing mortality profiles. This also calls into question the practice of forecasting observed age-specific rates of decline in disability \citep{Manton2006,Khaw1999}. Health economists refer to a similar 'red herring' argument, namely that medical costs are more closely associated with time-to-death than with chronological age. As a result, health care cost projections based on a chronological rather than thanatological age pattern of mortality are artificially inflated \citep{Zweifel1999,Geue2014}.

Instead, we argue that the better way to measure changes in health or disability is from a cohort perspective \citep{Manton2000,Manton2008,Christensen2013}. \citet{Manton2000}, for instance, found large differences between period and cohort estimates of active life expectancy (ALE). ALE at ages 65 and 85 was between 1.6 and 2.6 times larger in the cohort perspective than for similar period estimates, and the expected years of life disabled were smaller. Additionally, they uncovered larger differences between the cohort and period perspectives for men than women, which they attribute to differences in disability transition rates between the sexes. We theorize that some of these larger differences might also be attributable to larger mortality reduction among men.

Several studies have looked at the macro relationship between overall mortality levels and sex differences in HLE. At higher levels of life expectancy, female advantage in healthy life expectancy diminishes, or even reverses into male advantage \citep{vanOyen2013}. Meanwhile, the larger the proportional female advantage in longevity, the larger the female excess in the proportion of life in poor health \citep{Luy2014}. That mortality levels and disability prevalence are related is perhaps not surprising. As our example illustrates, differences in the underlying mortality lead to differences in the age profile of disability. Additionally, although the association between the severity of chronic conditions and poor health was found to be similar for men and women, morbidity prevalence rates are generally higher among women, particularly for arthritis and chronic pain \citep{Case2005}. It would be worthwhile to investigate whether there might not only be differences in the composition of chronic conditions between the sexes, but whether the underlying morbidity process itself might differ between the sexes in its chronological versus thanatological axis \citep{riffe2015ttd}. 

In reality, not all end-of-life health conditions are exclusive functions of
time-to-death, but morbidity often varies as a function of
both chronological and thanatological age, and it is best to express morbidity
as a function of both age and time-to-death, $g(a,y)$. There is great variety in
the temporal variation of late-life health conditions \citep{riffe2015ttd}.
Further, the function $g(a,y)$ changes over time and it is not fixed as in our examples. Nevertheless, the distortions demonstrated are likely to arise in everyday practice when comparing health trends over age between populations
and over time, since many health conditions appear to show strong time-to-death
components. Trends in mortality may offset or amplify changes in morbidity.
Therefore, in order to separate effects, more careful measurements are required
than is typically the case. 

\section{Appendix}


\begin{figure}
\begin{adjustwidth}{-1.5cm}{}
	\centering
	\includegraphics[scale=.8]{Decomp_3x3_rel.pdf}
	\caption{Decomp results...}
	\label{fig:Fig_Decomp_3x3_rel}
\end{adjustwidth}
\end{figure}



\newpage%% this forces a new page, so that things end up a bit more where you think they should
\bibliographystyle{chicago}
\bibliography{referencesTim}

\end{document}