% manuscript, to be drafted

%%This is a very basic article template.
%%There is just one section and two subsections.
%\documentclass[12pt,oneside,a4paper,doublespacing]{article} % for submission
\documentclass[11pt,oneside,a4paper]{article} % for sharing

\usepackage{appendix}
\usepackage{amsmath}
\usepackage{caption}
\usepackage{placeins}
\usepackage{graphicx}
\usepackage{subcaption}
%\usepackage{subfig}
\usepackage{longtable}
\usepackage{setspace}
%\usepackage{tikz}
\usepackage{booktabs}
\usepackage{tabularx}
\usepackage{xcolor,colortbl}
\usepackage{chngpage}
%\usepackage[active,tightpage]{preview}
\usepackage{natbib}
\bibpunct{(}{)}{,}{a}{}{;} 
\usepackage{url}
\usepackage{nth}
\usepackage{authblk}
\usepackage[most]{tcolorbox}
%\usepackage{hyperref}
%\usepackage{color}
%\usepackage{fontspec}
%\usepackage{pdfsync}
\usepackage[normalem]{ulem}
\usepackage{amsfonts}
%\renewcommand{\listtablename}{List of Appendix Tables}
%\newcolumntype{C}[1]{>{\centering\let\newline\\\arraybackslash\hspace{0pt}}m{#1}}
%\newcolumntype{L}[1]{>{\raggedright\let\newline\\\arraybackslash\hspace{0pt}}m{#1}}
% working on this need to concatenate file name based on sex and variable name
%\newcommand\Cell[1]{{\raisebox{-0.05in}{\includegraphics[height=.2in,width=.2in]{Figures/ColorCodes/\expandafter#1}}}}  

%%%%%%%%%%%%%%%%%%%%%%%%%%%%%%%%%%%%%%%%%%%%%%%%%%%%%%%%%%%%%%%%%%%%%%%%%%%%%
% setting color to letters affects spacing. Here's a hack I found here:
% http://tex.stackexchange.com/questions/212736/change-letter-colour-without-losing-letter-spacing
%\DeclareRobustCommand{\spacedallcaps}[1]{\MakeUppercase{\textsc{#1}}} % all
% caps with better spacing

%\colorlet{RED}{red}
%\colorlet{BLUE}{b}
%\colorlet{rd}{red}
%\colorlet{bl}{blue}

%%%%%%%%%%%%%%%%%%%%%%%%%%%%%%%%%%%%%%%%%%%%%%%%%%%%%%%%%%%%%%%%%%%%%%%%%%%%%%

\newcommand\ackn[1]{%
  \begingroup
  \renewcommand\thefootnote{}\footnote{#1}%
  \addtocounter{footnote}{-1}%
  \endgroup
}
%\newcommand\vt[1]{\textcolor{rd}{#1}}
%\newcommand\eg[1]{\textcolor{bl}{#1}}

%\newcommand\tg[1]{\includegraphics[scale=.5]{Figures/triadtable/triad#1.pdf}}
%\newcommand\tgh[1]{\raisebox{-.25\height}{\includegraphics[scale=.3]{Figures/triadtable/triad#1.pdf}}}

\defcitealias{HMD}{HMD}
\newcommand{\dd}{\; \mathrm{d}}
\newcommand{\tc}{\quad\quad\text{,}}
\newcommand{\tp}{\quad\quad\text{.}}
% junk for longtable caption
\AtBeginEnvironment{longtable}{\linespread{1}\selectfont}
\setlength{\LTcapwidth}{\linewidth}

%%%%%%%%%%%%%%%%%%%%%%%%%%%%%%%
\begin{document}

\title{Accounting for temporal variation in morbidity measurement
and projections}

\author[1]{Alyson van Raalte\thanks{vanraalte@demogr.mpg.de}}
\author[1]{Tim Riffe}
%\author[3]{John MacInnes}
\affil[1]{Max Planck Institute for Demographic Research}
%\affil[2]{Department of Demography, University of California, Berkeley}
%\affil[3]{School of Social and Political Science, University of Edinburgh}

%\author{[Authors]}

\maketitle

\begin{abstract}
This is important stuff!
\end{abstract}

$G$ is a bad health condition that varies as a function of time to death,
$y$ and not as a function of chronological age, $a$. However, there will still be an
apparent age function, $g'(a)$, given that $g(y)$ is regular and mortality is
sort of stable, but not really. $g'(a)$, in this case, is an aggregate based on
both mortality and the real underlying time-to-death process:
\begin{align}
g'(a) &= \frac{\int _0^\omega g(y) N(a,y) \dd y}{N(a)} \\
      &= \frac{\int _0^\omega g(y) N(a)
      \mu(a+y)\frac{\ell(a+y)}{\ell(a)}\dd y}{N(a)}\\
      &= \int _0^\omega g(y) f(y|a)\dd y
\end{align}
a little excercise we still need to do: find the $g'(a)$ that belongs to $g(y)$
in our canned example. It will be different for males and females because they
have different mortality schedules. In this case, we can make the healthy l;ife
expectancy function be based on mortality and $g'(a)$ and see what would be the
prediction if $g'(a)$ is held constant and we induce mortality improvement. The
answer is that mortality improvement will appear to increase the proportion of
remaining life expectancy that is unhealthy: also the absolute years spent
unhealthy, but the change in sex gap is maybe ambiguous (gotta check, maybe
not), depending on the changes induced.


brief interlude \ldots.
This is a simple caes of $g(y)$, but in reality morbidity often varies as a
function of both chronological and thanatological age, and we ought to have a
function $g(a,y)$.

\section{temp section, out of sync, just for latex}

Given the numbers from Figures X, there are various methods that one can use to
calculate period and cohort lifetables. For the sake of reproducibility for our
toy example, we describe steps as follows, firs tfor periods, then for cohorts.
\subsection{Period quantities}
We use event exposure lifetables, though it would be possible to jump straight
to death quotients from the given Lexis diagram.
Exposures for age $x$ in year $t$, $E(x,t)$, are calculated, per the HMD Methods
Protocol (cite) as:
\begin{equation}
E(x,t) = \frac{C(x,t) + C(x,t+1)}{2} + {D_L(x,t) - D_U(x,t)}{6} \tc
\end{equation}
where $C(x,t)$ is the census population in age interval $[x,x+1)]$ on January 1
of year $t$, $D_L(x,t)$ are deaths in the lower Lexis triangle of age $x$ in
year $t$, i.e., belonging to the cohort born in the year interval $[t-x,t-x+1)$.
$D_U(x,t)$ are deaths in the upper Lexis triangle of age $x$ in
year $t$, i.e., belonging to the cohort born in the year interval
$[t-x-1,t-x)$. All standard period lifetable steps are followed from the HMD
Methods Protocol, with the exception of the $a(x)$ assumption. The HMD assumes period $a(x)$
values of $\frac{1}{2}$. Instead, we apply the following formula:
\begin{equation}
a(x,t) = \frac{D_L(x,t)\frac{1}{3} + D_U(x,t)\frac{2}{3}}{D_L(x,t) + D_U(x,t)}
\tp
\end{equation}
We then proceed to calculate all columns through $e(x)$. 

The average value of the unhealthy condition $G$ at age $x$in year $t$, $g(x,t)$
is calculated as follows. We first convert counts unhealthy on birthdays to
proportions, and then take the arithmetic average of the proportion unhealthy at
age $x$ and age $x+1$. Expectancies are then calculated as follows:
\begin{align}
e(0,t) =&\sum _0^2 L(x,t) \\
e_U(0,t) =&\sum _0^2 L(x,t) g(x,t) \\
e_H(0,t) =& e(0,t) - e_U(0,t) \tc
\end{align}
where $e(0,t)$ is the life expectancy at birth in year $t$, $e_U(0,t)$ is
unhealthy life expectancy, and $e_H(0,t)$ is healthy life expectancy.

\subsection{Cohort quantities}
For cohorts we procede directly from within-cohort age interval survival
probabilities, $p(x,c)$, as follows:
\begin{equation}
p(x,c) = \frac{B(x+1,c)}{B(x,c)} \tc
\end{equation}
where $B$ are birthdays (horizontal counts), $x$ indexes the lower age interval
bound, and $c$ is the cohort born in the year interval $[c,c+1)]$. Starting with
a radix of 1, we calculate $l(x,c)$ as:
\begin{equation}
l(x,c) = \prod_0^x p(a,c)
\end{equation}
$L(x,c)$ is calculated as the arithmetic average of lower $l(x,c)$ and
$l(x+1,c)$. The average value of $G$ for the AC parallelogram is calculated
similarly as for period squares, except that we take the arithmetic average of
the proportions unhealthy at birthday $x$ and $x+a$ within the same cohort.
Expectancies are then calculated in the same way.

























\end{document}